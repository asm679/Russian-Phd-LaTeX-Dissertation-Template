
\begin{center}
	\textbf{Актуальность темы исследования, степень её разработанности}
\end{center}

{\actuality}
Задачи передач координат, построения геодезических сетей возникли столетия назад и, как правило, решались применением таких методик как линейно-угловые и астрономо-геодезические методы измерений. Однако, являясь трудозатратными, они требуют большого количества времени, как на выполнение полевых работ, так и для камеральной обработки получаемых данных.

Спутниковые методы измерений стали неотъемлемой частью производства геодезических работ в части построения геодезических сетей и координирования отдельных точек. В ходе своего развития спутниковые методы показали ряд преимуществ перед классическими геодезическими методами и на сегодняшний день среди них можно выделить следующие: построение геодезических сетей протяженных объектов; повышение точности координатных определений; сокращение временных и вычислительных затрат на полевые и камеральные работы. Однако, как правило, точность позиционирования в закрытых или полузакрытых областях (например, лесной местности или городской застройка) ниже, чем на открытой местности. Область применения спутниковых методов для нужд геодезии обширна и их применяют, начиная от одиночного позиционирования и заканчивая построением геодезических сетей.

До недавних пор для создания геодезических сетей, координирования отдельных точек, решения геодинамических задач применяли относительные методы спутниковых координатных определений. Но на сегодняшний день, благодаря активному развитию вычислительных и телекоммуникационных технологий, появилась возможность использования высокоточных эфемерид, алгоритмов с задержкой и данных, полученных с использование пунктов глобального распределения ITRF. В следствие этого стало возможным использование не только освоенных классических, но и более современных методов обработки спутниковых данных.

Чуть больше двадцати лет назад в NASA (Nasa Jet Propulsion Laboratory) был разработан метод высокоточных координатных определений или по-другому Precise Point Positioning (далее «PPP»). Однако, термин метод высокоточных координатных определений еще не устоялся в отечественной геодезической терминологии, а какая-либо регламентирующая информация в зарубежном и российском научном сообществах в настоящее время отсутствует. В следствии этого принято сохранять аббревиатуру РРР. 

{\progress}
% Этот раздел должен быть отдельным структурным элементом по
% ГОСТ, но он, как правило, включается в описание актуальности
% темы. Нужен он отдельным структурынм элемементом или нет ---
% смотрите другие диссертации вашего совета, скорее всего не нужен.
\textit{Степень разработанности} темы определяется исследованием научных публикаций и трудов в области использования РРР-алгоритмов. Авторами трудов, играющих важнейшую роль, являются следующие отечественные и зарубежные ученые: Виноградов А.В., Антонович К.М., Калинков В.В., Кафтан В.И., Липатников А.А., Мельников А.Ю., Тихонов А.Д., Шевчук С.О., Щербаков А.С., Устинов А.В., Abou-Callala M., Rabah M, Kallop M., Ebner R., Featherstone W.E., Kouba J., Heroux P., Kowalchuk K.

В работах \cite{mak01,mak02} отражены результаты исследования получаемых точностей при использовании РРР-алгоритма в зависимости от вариаций различных факторов. Исследование возможности применения РРР-алгоритма для построения геодезических сетей различных объектов и различного назначения приведены в работах \cite{mak03,mak09}. Из них следует, что точность обработки по РРР-алгоритму достигает сантиметрового и субсантиметрового уровня при продолжительности измерений не менее двух часов. Основываясь на работах \cite{mak05,mak08,src43,src44} можно утверждать, что точность РРР-алгоритма при одиночном позиционировании составляет несколько сантиметров.

Для реализации РРР-алгоритма требуется дополнительная информация: высокоточные эфемериды, локальные модели ионосферы и тропосферы. Помимо этого вводятся поправки за движение литосферных плит, приливы и отливы; а также учитывается ряд иных факторов, оказывающих влияние на точность передачи координат по РРР-алгоритму обработки спутниковых данных.  Метод не является разностным и не обладает свойством компенсации односторонне действующих ошибок, поэтому \textbf{надежное определение его современных возможностей является актуальной задачей}.

Сегодня развито немалое количество способов обработки данных с использованием РРР-алгоритма. В качестве интернет-сервисов могут быть названы следующие: Automatic Precise Points Positioning(APPS), Canadian Spatial Reference System Precise Points Positioning(CSRS-PPP), GNSS Analysis and Positioning Software (GAPS), magic-PPP (magic GNSS), Trimble-RTX и ряд других. В свою очередь среди программных продуктов можно выделить: GPS Toolkit, Bernese, Waypoint GrafNET, GIPSY-OASIS, RTKLIB. 

{\aim} \textbf{диссертационной работы} является совершенствование методики передачи координат в широком диапазоне расстояний при использовании современных методов обработки спутниковых данных, а также исследование различных факторов, влияющих на точность определения координат по РРР-алгоритму.

{\tasks} \textbf{диссертационной работы}:
\begin{enumerate}[beginpenalty=10000] % https://tex.stackexchange.com/a/476052/104425
	\item Теоретическое и экспериментальное обоснование возможности использования РРР-алгоритма для построения геодезических сетей в различных диапазонах расстояний.
	\item Разработка методики оценки точности при использовании РРР-алгоритма.
	\item Теоретические и экспериментальные исследования с целью выявления факторов, влияющих на точность обработки данных по РРР-алгоритму.
\end{enumerate}

{\theoretical}

В первой главе теоретически рассмотрены факторы, влияющие на точность передачи координат с использованием спутниковых методов. Помимо этого, приводится классификация РРР-методов обработки спутниковых данных.

В начале второй главы описаны современные методы обработки с использованием РРР-алгоритма. Произведено обобщение и сравнение существующих подходов на примере коммерческих и свободных программных продуктов, а также интернет-сервисов.

{\influence} диссертационной работы определяется разработанной усовершенствованной технологией построения геодезических сетей. Выполнено исследование точности при обработке данных одиночного позиционирования и оценены различные факторы, влияющие на точность определения координат: продолжительность измерений, количество используемых спутников, дискретность данных (частота записи), выбор подходов к обработке (с использованием интернет-сервисов или программных средств).

{\novelty} диссертационного исследования заключается в следующем:
\begin{enumerate}[beginpenalty=10000] % https://tex.stackexchange.com/a/476052/104425
	\item Проведено обширное исследование, в результате которого обоснована возможность получения координат, в том числе высокоточных \underline{\textbf{(дать конкретику)}}.
	\item Данное исследование с одной стороны позволяет создавать геодезические сети с уменьшением временных затрат на полевые и камеральные \textbf{работы}, а с другой стороны \textbf{повысить} точность определения координат пунктов сети.
	\item Предложена методика создания высокоточных геодезических сетей с использованием PPP-алгоритма обработки спутниковых данных.
\end{enumerate}

{\methods}

В процессе написания диссертации проанализировано несколько десятков научных источников как отечественных, так и зарубежных. На их основании были поставлены основные этапы диссертационного исследования.

{\defpositions}
\begin{enumerate}[beginpenalty=10000] % https://tex.stackexchange.com/a/476052/104425
	\item Результаты исследования точности определения координат при использовании РРР-алгоритма.
	\item Возможность получения координат с помощью современных программных средств.
	\item Методика применения РРР-алгоритма при создании геодезических сетей.
\end{enumerate}

{\pasport}
Основные положения диссертации соответствуют паспорту научной специальности 1.6.22 геодезия в части пунктов: 3, 4, 5.

\begin{enumerate}
	\setcounter{enumi}{2}
	\item Создание и развитие геодезической координатно-временной основы различного назначения с использованием геодезических, астрономических, гравиметрических и других (космических, наземных, подземных и подводных) методов измерений; оценка их стабильности и характера изменений, вопросы проектирования и оптимизации. Разработка и развитие теорий построения и реализации координатных, высотных и гравиметрических систем отсчета.
	\item Геодезические (глобальные) навигационные спутниковые системы (ГНСС) и технологии. Формирование активной координатно-временной инфраструктуры на основе ГНСС. Методы и технологии высокоточного определения местоположения и навигации по сигналам спутниковых навигационных систем. Геодезические системы наземного, морского и космического базирования для определения местоположения и навигации подвижных объектов геопространства. Многосистемные и высокоскоростные (высокочастотные) ГНСС приложения. ГНСС рефлектометрия. \looseness=-1
	\item Теория и практика математической обработки результатов геодезических измерений и информационное обеспечение геодезических работ.
\end{enumerate}

Очевидно, что все пункты паспорта научной специальности 1.6.22 «Геодезия, подтверждают актуальность темы диссертационного исследования.

\begin{center}
	{\reliaprobation}
\end{center}

Теоретические и практические изыскания доказывают, что тема диссертации актуальна и соответствует заявленной.

Основные положения и полученные результаты научных исследований были доложены и обсуждены в ряде как российских, так и международных конференций.  Материалы, оборудование и ресурсосберегающие технологии, Могилев, 2021; Материалы, оборудование и ресурсосберегающие технологии, Могилев, 2022; Материалы, оборудование и ресурсосберегающие технологии, Могилев, 2023; Материалы, оборудование и ресурсосберегающие технологии, Могилев, 2024;  Конференция, посвященная 125-летию Российского Университета Транспорта, Москва, Россия, 2021. Конференция DiEarth 2021: Международная научно-исследовательская конференция по перспективные исследования Земли; геодезия, геоинформатика, картография, землеустройство и кадастры. Пространственные данные: наука и технологии, Москва, Россия, 2024.

Основные результаты диссертационного исследования внедрены в учебный процесс в Федеральном государственном автономном образовательном учреждении высшего образования Российском университете транспорта при изучении дисциплин: «Высшая геодезия» и «Спутниковые навигационные системы в кадастре», а также в учебный процесс в федеральном государственном бюджетном образовательном учреждении высшего образования Государственном Университете по Землеустройству при изучении дисциплин: «Высшая геодезия» (модуль «Спутниковые системы в геодезии» и «спутниковые технологии в геодезии»).

%{\publications} \par Основные результаты по теме диссертации изложены в 10 печатных изданиях, 4 из которых изданы в журналах, рекомендованных ВАК, 6 –– в %сборниках трудов и конференций.%
%
%
%\ifnumequal{\value{bibliosel}}{0}%
%{%%% Встроенная реализация с загрузкой файла через движок bibtex8. (При желании, внутри можно использовать обычные ссылки, наподобие %`\cite{vakbib1,vakbib2}`).
%    {\publications} \\Основные результаты по теме диссертации изложены %
%    в~XX~печатных изданиях, %
%    X из которых изданы в журналах, рекомендованных ВАК, %
%    X "--- в тезисах докладов. %
%}%
%{%%% Реализация пакетом biblatex через движок biber
%    \begin{refsection}[bl-author, bl-registered]%
%         Это refsection=1.%
%         Процитированные здесь работы:%
%          * подсчитываются, для автоматического составления фразы "Основные результаты ..."%
%          * попадают в авторскую библиографию, при usefootcite==0 и стиле `\insertbiblioauthor` или `\insertbiblioauthorgrouped`%
%          * нумеруются там в зависимости от порядка команд `\printbibliography` в этом разделе.%
%          * при использовании `\insertbiblioauthorgrouped`, порядок команд `\printbibliography` в нём должен быть тем же (см. biblio/biblatex.tex)%
%        
%         Невидимый библиографический список для подсчёта количества публикаций:%
%
%{\publications} \par Основные результаты
%
{%
\begin{refsection}[bl-author, bl-registered]%
\nocite{*}%
\printbibliography[heading=nobibheading, section=1, env=countauthorvak,          keyword=biblioauthorvak]%
\printbibliography[heading=nobibheading, section=1, env=countauthorconf,         keyword=biblioauthorconf]%
\printbibliography[heading=nobibheading, section=1, env=countauthor,             keyword=biblioauthor]%
%\printbibliography[heading=nobibheading, section=1, env=countauthorwos,          keyword=biblioauthorwos]%
%\printbibliography[heading=nobibheading, section=1, env=countauthorscopus,       keyword=biblioauthorscopus]%
%\printbibliography[heading=nobibheading, section=1, env=countauthorother,        keyword=biblioauthorother]%
%\printbibliography[heading=nobibheading, section=1, env=countregistered,         keyword=biblioregistered]%
%\printbibliography[heading=nobibheading, section=1, env=countauthorpatent,       keyword=biblioauthorpatent]%
%\printbibliography[heading=nobibheading, section=1, env=countauthorprogram,      keyword=biblioauthorprogram]%
%\printbibliography[heading=nobibheading, section=1, env=countauthorvakscopuswos, filter=vakscopuswos]%
%\printbibliography[heading=nobibheading, section=1, env=countauthorscopuswos,    filter=scopuswos]%
%        
%        \nocite{*}%
%        
        {\publications} \par Основные результаты по теме диссертации изложены в \arabic{citeauthor} печатных изданиях, %
        \arabic{citeauthorvak} из которых изданы в журналах, рекомендованных ВАК\sloppy%
        \ifnum \value{citeauthorscopuswos}>0%
            , \arabic{citeauthorscopuswos} "--- в~периодических научных журналах, индексируемых Web of~Science и Scopus\sloppy%
        \fi%
        \ifnum \value{citeauthorconf}>0%
            , \arabic{citeauthorconf} "--- в сборниках трудов и конференций.%
        \else%
            .%
        \fi%
        \ifnum \value{citeregistered}=1%
            \ifnum \value{citeauthorpatent}=1%
                Зарегистрирован \arabic{citeauthorpatent} патент.%
            \fi%
            \ifnum \value{citeauthorprogram}=1%
                Зарегистрирована \arabic{citeauthorprogram} программа для ЭВМ.%
            \fi%
        \fi%
        \ifnum \value{citeregistered}>1%
            Зарегистрированы\ %
            \ifnum \value{citeauthorpatent}>0%
            \formbytotal{citeauthorpatent}{патент}{}{а}{}\sloppy%
            \ifnum \value{citeauthorprogram}=0 . \else \ и~\fi%
            \fi%
            \ifnum \value{citeauthorprogram}>0%
            \formbytotal{citeauthorprogram}{программ}{а}{ы}{} для ЭВМ.%
            \fi%
        \fi%
%         К публикациям, в которых излагаются основные научные результаты диссертации на соискание учёной
%         степени, в рецензируемых изданиях приравниваются патенты на изобретения, патенты (свидетельства) на
%         полезную модель, патенты на промышленный образец, патенты на селекционные достижения, свидетельства
%         на программу для электронных вычислительных машин, базу данных, топологию интегральных микросхем,
%         зарегистрированные в установленном порядке.(в ред. Постановления Правительства РФ от 21.04.2016 N 335)
    \end{refsection}%
%    \begin{refsection}[bl-author, bl-registered]%
%         Это refsection=2.
%         Процитированные здесь работы:
%          * попадают в авторскую библиографию, при usefootcite==0 и стиле `\insertbiblioauthorimportant`.
%          * ни на что не влияют в противном случае
        %\nocite{vakbib2}%vak
        %\nocite{patbib1}%patent
        %\nocite{progbib1}%program
        %\nocite{bib1}%other
        %\nocite{confbib1}%conf
        %\nocite{mak01, mak02, mak03, mak04, mak05, mak06, mak07, mak08, mak09, mak10, mak99}%
%    \end{refsection}%
%
%         Всё, что вне этих двух refsection, это refsection=0,
%          * для диссертации - это нормальные ссылки, попадающие в обычную библиографию
%          * для автореферата:
%             * при usefootcite==0, ссылка корректно сработает только для источника из `external.bib`. Для своих работ --- напечатает "[0]" (и даже Warning не вылезет).
%             * при usefootcite==1, ссылка сработает нормально. В авторской библиографии будут только процитированные в refsection=0 работы.
}%

{\contribution}

Содержание диссертации и основные положения подтверждают персональный вклад.

