\chapter{ИССЛЕДОВАНИЕ ТОЧНОСТИ ПОСТРОЕНИЯ ГЕОДЕЗИЧЕСКИХ СЕТЕЙ ПРИ УРАВНИВАНИИ СЕТЕЙ В ДИАПАЗОНАХ РАССТОЯНИЙ 5-700 КМ}\label{ch:ch3}

\section{Исследование точности определения координат в диапазоне расстояний 5-100 км}\label{sec:ch3/sect1}

\subsection{Исходные данные для спутниковой сети в диапазоне расстояний 0-100 км}\label{subsec:ch3/sec1/sub1}

В параграфе рассматривается геодезическая сеть, состоящая из 5 пунктов. Пункты равномерно расположены в диапазоне расстояний 5-100 км. RINEX-файлы с пунктов сети были получены с сайта базовых станций EFT COORS \cite{src55}. На рисунке \cref{fig:pic25}  приведена схема рассматриваемой сети.


\begin{figure}[!h]
	\centering
	\includegraphics[width=0.99\linewidth]{images/pic25}
	\caption{Cпутниковая сеть в диапазоне расстояний 0-100 км }
	\label{fig:pic25}
\end{figure}

В качестве исходного пункта был выбран пункт LOBN, отмеченный на рисунке треугольным маркером. Определяемые пункты (RUZA, DMTR, ZHDR, MSKV2) --- отмечены круглым маркером.

Для исключения ошибок центрирования приемники не переставлялись и высоты антенны не вводились. В этом случае происходит определение координат фазовых центров спутниковых антенн. В таблице \cref{tab:tab24} приведены координаты пунктов геодезической сети.

\begin{table} [htbp]
	\centering\small
	\begin{threeparttable}% выравнивание подписи по границам таблицы
	\captionof{table}{Эталонные координаты геодезических пунктов в системе координат ITRF 2014 }\label{tab:tab24}{
	\begin{tabular}{|c|c|c|c|}
		\hline
		\textbf{Метка (№)}			& \textbf{X, м}		& \textbf{Y, м}		& \textbf{Z, м}		\\ \hline
		LOBN $(1)$					& $2837136,604$		& $2165621,902$		& $5268527,919$		\\ \hline
		RUZA $(2)$					& $2907267,745$		& $2127203,200$		& $5246025,953$		\\ \hline
		DMTR $(3)$					& $2811450,988$		& $2158239,108$		& $5285093,057$		\\ \hline
		MSK2 $(4)$					& $2847083,993$		& $2191776,459$		& $5252448,474$		\\ \hline
		ZHDR $(5)$					& $2834298,462$		& $2216036,180$		& $5249102,708$		\\ \hline
	\end{tabular}
	}
	\end{threeparttable}
\end{table}

Основываясь на таблице \cref{tab:tab24} были получены эталонные координаты, приведенные в таблице \cref{tab:tab25}.

\begin{table} [htbp]
	\centering\small
	\begin{threeparttable}% выравнивание подписи по границам таблицы
		\captionof{table}{Эталонные приращения координат геодезических пунктов в системе координат ITRF 2014 }\label{tab:tab25}
%		\renewcommand{\arraystretch}{1.6}
		\setlength{\tabcolsep}{20pt}
		\sisetup{
			table-figures-integer = 6,
			table-figures-decimal = 3
		}
		
			\begin{tabular}{ l
				             S[table-number-alignment = center]
				             S[table-number-alignment = center]
				             S[table-number-alignment = center] }
			
				
				\toprule
				\textbf{Вектор)}			& \textbf{dX, м}	& \textbf{dY, м}	& \textbf{dZ, м}	\\ \midrule
				DMTR-LOBN					&  25685,616		&   7382,794		& -16565,138		\\ %\hline
				DMTR-MSK2					&  35633,005		&  33537,351		& -32644,583		\\ %\hline
				DMTR-RUZA					&  95816,757		& -31035,908		& -39067,104		\\ %\hline
				DMTR-ZHDR					&  22847,474		&  57797,072		& -35990,349		\\ %\hline
				LOBN-MSK2					&   9947,389		&  26154,557		& -16079,445		\\ %\hline
				LOBN-RUZA					&  70131,141		& -38418,702		& -22501,966		\\ %\hline
				LOBN-ZHDR					&  -2838,142		&  50414,278		& -19425,211		\\ %\hline
				MSK2-RUZA					&  60183,752		& -64573,259		&  -6422,521		\\ %\hline
				MSK2-ZHDR					& -12785,531		&  24259,721		&  -3345,766		\\ %\hline
				RUZA-ZHDR					& -72969,283		&  88832,980		&   3076,755		\\ \bottomrule
			\end{tabular}
		
	\end{threeparttable}
\end{table}

В инструкции \cite{src10,src19} сказано, что при построении геодезических сетей все линии сети должны быть определены независимо. Это требование важно, т.к. включение зависимых линий в обработку сети не позволяет правильно произвести оценку точности и надежности измерений. В случае спутниковых измерений независимыми измерениями считаются вектора, определенные в различный временной диапазон.


\subsection{Уравнивание спутниковой сети в статике}\label{subsec:ch3/sec1/sub2}

Согласно главе \cref{ch:ch2} минимальное время для получения точного решения при использовании одного приемника составляет не менее двух часов. Для формирования независимых измерений, сеансы проводились с 12 часовыми перерывами.

В таблице \cref{tab:tab26} приведены сеансы измерений и определяемые в них вектора:

\begin{table} [htbp]
	\centering\small
	\begin{threeparttable}% выравнивание подписи по границам таблицы
		\captionof{table}{Распределение векторов по сеансам измерений }\label{tab:tab26}
		%		\renewcommand{\arraystretch}{1.6}
		\setlength{\tabcolsep}{20pt}
		\sisetup{
			table-figures-integer = 2,
			table-figures-decimal = 0
		}
		
		\begin{tabular}{ S c }
			\toprule
			\textbf{Сеанс}	& \textbf{Вектора}								\\ \midrule
			1				&  LOBN-RUZA, DMTR-LOBN, DMTR-ZHDR, MSK2-ZHDR	\\ %\hline
			2				&  LOBN-ZHDR, ZHDR-RUZA, RUZA-MSK2				\\ %\hline
			3				&  RUZA-DMTR, DMTR-MSK2, LOBN-MSK2 				\\ \bottomrule
		\end{tabular}	
	\end{threeparttable}
\end{table}

На рисунке \cref{fig:pic26} приведены независимые сеансы измерений.


\begin{figure}[!h]
	\centering
	\includegraphics[width=0.99\linewidth]{images/pic26}
	\caption{Независимые сеансы измерений }
	\label{fig:pic26}
\end{figure}

Исходный пункт: LOBN, остальные --- определяемые. Продолжительность сеансов измерения составляла $1, 2, 3$ И $4$~часа, каждый раз формируя независимые вектора. Сначала была выполнена обработка одночасовых интервалов в статике с использованием программного обеспечения КРЕДО ГНСС. В таблице \cref{tab:tab27} приведены	приращения	координат,	полученных	при	обработке трёх независимых блоков, указанных выше.

\begin{table} [htbp]
	\centering\small
	\begin{threeparttable}% выравнивание подписи по границам таблицы
		\captionof{table}{приращения координат пунктов, полученные при обработке данных в статике при 1 часе }\label{tab:tab27}
		\setlength{\tabcolsep}{20pt}
		\sisetup{
			table-figures-integer = 6,
			table-figures-decimal = 3
		}
		
		\begin{tabular}{ l
				S[table-number-alignment = center]
				S[table-number-alignment = center]
				S[table-number-alignment = center] }
			
			
			\toprule
			\textbf{Вектор)}			& \textbf{ΔX, м}	& \textbf{ΔY, м}	& \textbf{ΔZ, м}	\\ \midrule
			LOBN-RUZA					&  70131,124		& -38418,657		& -22501,925		\\ %\hline
			DMTR-LOBN					&  25685,641		&   7382,792		& -16565,180		\\ %\hline
			DMTR-ZHDR					&  22847,524		&  57797,043		& -35990,348		\\ %\hline
			MSK2-ZHDR					& -12785,551		&  24259,785		&  -3345,774		\\ %\hline
			LOBN-ZHDR					&  -2838,158		&  50414,134		& -19425,163		\\ %\hline
			ZHDR-RUZA					&  72969,257		& -88832,959		&  -3076,725		\\ %\hline
			RUZA-MSK2					& -60183,863		&  64573,354		&   6422,525		\\ %\hline
			RUZA-DMTR					& -95816,737		&  31035,897		&  39067,096		\\ %\hline
			DMTR-MSK2					&  35632,970		&  33537,301		& -32644,604		\\ %\hline
			LOBN-MSK2					&   9947,381		&  26154,528		& -16079,454		\\ \bottomrule
		\end{tabular}
		
	\end{threeparttable}
\end{table}

Уравнивание упрощенным классическим способов можно записать в виде формулы~\cref{eq:eq3.1}

\begin{equation}
	\label{eq:eq3.1}
		V = A d + L,
\end{equation}
где: 
\begin{description}
	\item[$V$] -- поправка; 
	\item[$A$] -- исходная матрица, характеризующая параметр сети;
	\item[$d$] -- вектор поправок;
	\item[$L$] -- невязки.
\end{description}

Для начала происходит определение приближенных координат пунктов, указываемое в таблице~\cref{tab:tab28}:


\begin{table} [htbp]
	\centering\small
	\begin{threeparttable}% выравнивание подписи по границам таблицы
		\captionof{table}{приближенные координаты пунктов }\label{tab:tab28}
		\setlength{\tabcolsep}{20pt}
		\sisetup{
			table-figures-integer = 7,
			table-figures-decimal = 3
		}
		
		\begin{tabular}{ l
				S[table-number-alignment = center]
				S[table-number-alignment = center]
				S[table-number-alignment = center] }
			
			
			\toprule
			\textbf{Станция (№)}		& \textbf{X, м}		& \textbf{Y, м}		& \textbf{Z, м}	\\ \midrule
			LOBN (1)					&  2837136,604		&  2165621,902		&  5268527,919		\\ %\hline
			RUZA (2)					&  2907267,742		&  2127203,214		&  5246025,992		\\ %\hline
			DMTR (3)					&  2811451,038		&  2158239,136		&  5285093,053		\\ %\hline
			MSK2 (4)					&  2847083,944		&  2191776,359		&  5252448,498		\\ %\hline
			ZHDR (5)					&  2834298,496		&  2216036,339		&  5249102,679		\\ \bottomrule
		\end{tabular}
		
	\end{threeparttable}
\end{table}

В дальнейшем находятся невязки измерений, приведенные в таблице 29


\begin{table} [htbp]
	\centering\small
	\begin{threeparttable}% выравнивание подписи по границам таблицы
		\captionof{table}{таблица L }\label{tab:tab29}
		\setlength{\tabcolsep}{20pt}
		\sisetup{
			table-figures-integer = 6,
			table-figures-decimal = 3
		}
		
		\begin{tabular}{ l
				S[table-number-alignment = center]
				S[table-number-alignment = center]
				S[table-number-alignment = center] }
			
			
			\toprule
			\textbf{Вектор)}			& \textbf{ΔX, м}	& \textbf{ΔY, м}	& \textbf{ΔZ, м}	\\ \midrule
			LOBN-RUZA					&  		& 		& 		\\ %\hline
			DMTR-ZHDR					&  		&  		& 		\\ %\hline
			MSK2-ZHDR					& 		&  		& 		\\ %\hline
			LOBN-ZHDR					&  		&  		& 		\\ %\hline
			ZHDR-RUZA					&  		&  		& 		\\ %\hline
			RUZA-MSK2					& 		&  		& 		\\ %\hline
			RUZA-DMTR					& 		&  		& 		\\ %\hline
			DMTR-MSK2					&  		&  		& 		\\ %\hline
			LOBN-MSK2					&   	&  		& 		\\ \bottomrule
		\end{tabular}
		
	\end{threeparttable}
\end{table}





























\vfill
\noindent\rule{\textwidth}{.1pt}
\begin{center}				Конец освоенного текста					\end{center}
\noindent\rule{\textwidth}{1pt}
\newpage
\newpage


















Так размещается таблица:

\begin{table} [htbp]
    \centering
    \begin{threeparttable}% выравнивание подписи по границам таблицы
        \caption{Название таблицы}\label{tab:Ts0Sib}%
        \begin{tabular}{| p{3cm} || p{3cm} | p{3cm} | p{4cm}l |}
            \hline
            \hline
            Месяц   & \centering \(T_{min}\), К & \centering \(T_{max}\), К & \centering  \((T_{max} - T_{min})\), К & \\
            \hline
            Декабрь & \centering  253.575       & \centering  257.778       & \centering      4.203                  & \\
            Январь  & \centering  262.431       & \centering  263.214       & \centering      0.783                  & \\
            Февраль & \centering  261.184       & \centering  260.381       & \centering     \(-\)0.803              & \\
            \hline
            \hline
        \end{tabular}
    \end{threeparttable}
\end{table}

\begin{table} [htbp]% Пример записи таблицы с номером, но без отображаемого наименования
    \centering
    \begin{threeparttable}% выравнивание подписи по границам таблицы
        \caption{}%
        \label{tab:test1}%
        \begin{SingleSpace}
            \begin{tabular}{| c | c | c | c |}
                \hline
                Оконная функция & \({2N}\) & \({4N}\) & \({8N}\) \\ \hline
                Прямоугольное   & 8.72     & 8.77     & 8.77     \\ \hline
                Ханна           & 7.96     & 7.93     & 7.93     \\ \hline
                Хэмминга        & 8.72     & 8.77     & 8.77     \\ \hline
                Блэкмана        & 8.72     & 8.77     & 8.77     \\ \hline
            \end{tabular}%
        \end{SingleSpace}
    \end{threeparttable}
\end{table}

Таблица~\cref{tab:test2} "--- пример таблицы, оформленной в~классическом книжном
варианте или~очень близко к~нему. \mbox{ГОСТу} по~сути не~противоречит. Можно
ещё~улучшить представление, с~помощью пакета \verb|siunitx| или~подобного.

\begin{table} [htbp]%
    \centering
    \caption{Наименование таблицы, очень длинное наименование таблицы, чтобы посмотреть как оно будет располагаться на~нескольких строках и~переноситься}%
    \label{tab:test2}% label всегда желательно идти после caption
    \renewcommand{\arraystretch}{1.5}%% Увеличение расстояния между рядами, для улучшения восприятия.
    \begin{SingleSpace}
        \begin{tabular}{@{}@{\extracolsep{20pt}}llll@{}} %Вертикальные полосы не используются принципиально, как и лишние горизонтальные (допускается по ГОСТ 2.105 пункт 4.4.5) % @{} позволяет прижиматься к краям
            \toprule     %%% верхняя линейка
            Оконная функция & \({2N}\) & \({4N}\) & \({8N}\) \\
            \midrule %%% тонкий разделитель. Отделяет названия столбцов. Обязателен по ГОСТ 2.105 пункт 4.4.5
            Прямоугольное   & 8.72     & 8.77     & 8.77     \\
            Ханна           & 7.96     & 7.93     & 7.93     \\
            Хэмминга        & 8.72     & 8.77     & 8.77     \\
            Блэкмана        & 8.72     & 8.77     & 8.77     \\
            \bottomrule %%% нижняя линейка
        \end{tabular}%
    \end{SingleSpace}
\end{table}

\section{Таблица с многострочными ячейками и примечанием}

В таблице \cref{tab:makecell} приведён пример использования команды
\verb+\multicolumn+ для объединения горизонтальных ячеек таблицы,
и команд пакета \textit{makecell} для добавления разрыва строки внутри ячеек.
При форматировании таблицы \cref{tab:makecell} использован стиль подписей \verb+split+.
Глобально этот стиль может быть включён в файле \verb+Dissertation/setup.tex+ для диссертации и в
файле \verb+Synopsis/setup.tex+ для автореферата.
Однако такое оформление не~соответствует ГОСТ.

\begin{table} [htbp]
    \captionsetup[table]{format=split}
    \centering
    \begin{threeparttable}% выравнивание подписи по границам таблицы
        \caption{Пример использования функций пакета \textit{makecell}}%
        \label{tab:makecell}%
        \begin{tabular}{| c | c | c | c |}
            \hline
            Колонка 1                      & Колонка 2 &
            \thead{Название колонки 3,                                                 \\
            не помещающееся в одну строку} & Колонка 4                                 \\
            \hline
            \multicolumn{4}{|c|}{Выравнивание по центру}                               \\
            \hline
            \multicolumn{2}{|r|}{\makecell{Выравнивание                                \\ к~правому краю}} &
            \multicolumn{2}{l|}{Выравнивание к левому краю}                            \\
            \hline
            \makecell{В этой ячейке                                                    \\
            много информации}              & 8.72      & 8.55                   & 8.44 \\
            \cline{3-4}
            А в этой мало                  & 8.22      & \multicolumn{2}{c|}{5}        \\
            \hline
        \end{tabular}%
    \end{threeparttable}
\end{table}

Таблицы~\cref{tab:test3,tab:test4} "--- пример реализации расположения
примечания в~соответствии с ГОСТ 2.105. Каждый вариант со своими достоинствами
и~недостатками. Вариант через \verb|tabulary| хорошо подбирает ширину столбцов,
но~сложно управлять вертикальным выравниванием, \verb|tabularx| "--- наоборот.
\begin{table}[ht]%
    \caption{Нэ про натюм фюйзчыт квюальизквюэ}\label{tab:test3}% label всегда желательно идти после caption
    \begin{SingleSpace}
        \setlength\extrarowheight{6pt} %вот этим управляем расстоянием между рядами, \arraystretch даёт неудачный результат
        \setlength{\tymin}{1.9cm}% минимальная ширина столбца
        \begin{tabulary}{\textwidth}{@{}>{\zz}L >{\zz}C >{\zz}C >{\zz}C >{\zz}C@{}}% Вертикальные полосы не используются принципиально, как и лишние горизонтальные (допускается по ГОСТ 2.105 пункт 4.4.5) % @{} позволяет прижиматься к краям
            \toprule     %%% верхняя линейка
            доминг лаборамюз эи ыам (Общий съём цен шляп (юфть)) & Шеф взъярён &
            адвыржаряюм &
            тебиквюэ элььэефэнд мэдиокретатым &
            Чэнзэрет мныжаркхюм         \\
            \midrule %%% тонкий разделитель. Отделяет названия столбцов. Обязателен по ГОСТ 2.105 пункт 4.4.5
            Эй, жлоб! Где туз? Прячь юных съёмщиц в~шкаф Плюш изъят. Бьём чуждый цен хвощ! &
            \({\approx}\) &
            \({\approx}\) &
            \({\approx}\) &
            \( + \) \\
            Эх, чужак! Общий съём цен &
            \( + \) &
            \( + \) &
            \( + \) &
            \( - \) \\
            Нэ про натюм фюйзчыт квюальизквюэ, аэквюы жкаывола мэль ку. Ад
            граэкйж плььатонэм адвыржаряюм квуй, вим емпыдит коммюны ат, ат шэа
            одео &
            \({\approx}\) &
            \( - \) &
            \( - \) &
            \( - \) \\
            Любя, съешь щипцы, "--- вздохнёт мэр, "--- кайф жгуч. &
            \( - \) &
            \( + \) &
            \( + \) &
            \({\approx}\) \\
            Нэ про натюм фюйзчыт квюальизквюэ, аэквюы жкаывола мэль ку. Ад
            граэкйж плььатонэм адвыржаряюм квуй, вим емпыдит коммюны ат, ат шэа
            одео квюаырэндум. Вёртюты ажжынтиор эффикеэнди эож нэ. &
            \( + \) &
            \( - \) &
            \({\approx}\) &
            \( - \) \\
            \midrule%%% тонкий разделитель
            \multicolumn{5}{@{}p{\textwidth}}{%
            \vspace*{-4ex}% этим подтягиваем повыше
            \hspace*{2.5em}% абзацный отступ - требование ГОСТ 2.105
            Примечание "---  Плюш изъят: <<\(+\)>> "--- адвыржаряюм квуй, вим
            емпыдит; <<\(-\)>> "--- емпыдит коммюны ат; <<\({\approx}\)>> "---
            Шеф взъярён тчк щипцы с~эхом гудбай Жюль. Эй, жлоб! Где туз?
            Прячь юных съёмщиц в~шкаф. Экс-граф?
            }
            \\
            \bottomrule %%% нижняя линейка
        \end{tabulary}%
    \end{SingleSpace}
\end{table}

Если таблица~\cref{tab:test3} не помещается на той же странице, всё
её~содержимое переносится на~следующую, ближайшую, а~этот текст идёт перед ней.
\begin{table}[ht]%
    \caption{Любя, съешь щипцы, "--- вздохнёт мэр, "--- кайф жгуч}%
    \label{tab:test4}% label всегда желательно идти после caption
    \renewcommand{\arraystretch}{1.6}%% Увеличение расстояния между рядами, для улучшения восприятия.
    \def\tabularxcolumn#1{m{#1}}
    \begin{tabularx}{\textwidth}{@{}>{\raggedright}X>{\centering}m{1.9cm} >{\centering}m{1.9cm} >{\centering}m{1.9cm} >{\centering\arraybackslash}m{1.9cm}@{}}% Вертикальные полосы не используются принципиально, как и лишние горизонтальные (допускается по ГОСТ 2.105 пункт 4.4.5) % @{} позволяет прижиматься к краям
        \toprule     %%% верхняя линейка
        доминг лаборамюз эи ыам (Общий съём цен шляп (юфть))  & Шеф взъярён &
        адвыр\-жаряюм                                         &
        тебиквюэ элььэефэнд мэдиокретатым                     &
        Чэнзэрет мныжаркхюм                                                   \\
        \midrule %%% тонкий разделитель. Отделяет названия столбцов. Обязателен по ГОСТ 2.105 пункт 4.4.5
        Эй, жлоб! Где туз? Прячь юных съёмщиц в~шкаф Плюш изъят.
        Бьём чуждый цен хвощ!                                 &
        \({\approx}\)                                         &
        \({\approx}\)                                         &
        \({\approx}\)                                         &
        \( + \)                                                               \\
        Эх, чужак! Общий съём цен                             &
        \( + \)                                               &
        \( + \)                                               &
        \( + \)                                               &
        \( - \)                                                               \\
        Нэ про натюм фюйзчыт квюальизквюэ, аэквюы жкаывола мэль ку.
        Ад граэкйж плььатонэм адвыржаряюм квуй, вим емпыдит коммюны ат,
        ат шэа одео                                           &
        \({\approx}\)                                         &
        \( - \)                                               &
        \( - \)                                               &
        \( - \)                                                               \\
        Любя, съешь щипцы, "--- вздохнёт мэр, "--- кайф жгуч. &
        \( - \)                                               &
        \( + \)                                               &
        \( + \)                                               &
        \({\approx}\)                                                         \\
        Нэ про натюм фюйзчыт квюальизквюэ, аэквюы жкаывола мэль ку. Ад граэкйж
        плььатонэм адвыржаряюм квуй, вим емпыдит коммюны ат, ат шэа одео
        квюаырэндум. Вёртюты ажжынтиор эффикеэнди эож нэ.     &
        \( + \)                                               &
        \( - \)                                               &
        \({\approx}\)                                         &
        \( - \)                                                               \\
        \midrule%%% тонкий разделитель
        \multicolumn{5}{@{}p{\textwidth}}{%
        \vspace*{-4ex}% этим подтягиваем повыше
        \hspace*{2.5em}% абзацный отступ - требование ГОСТ 2.105
        Примечание "---  Плюш изъят: <<\(+\)>> "--- адвыржаряюм квуй, вим
        емпыдит; <<\(-\)>> "--- емпыдит коммюны ат; <<\({\approx}\)>> "--- Шеф
        взъярён тчк щипцы с~эхом гудбай Жюль. Эй, жлоб! Где туз? Прячь юных
        съёмщиц в~шкаф. Экс-граф?
        }
        \\
        \bottomrule %%% нижняя линейка
    \end{tabularx}%
\end{table}

\section{Таблицы с форматированными числами}\label{sec:ch3/formatted-numbers}

В таблицах \cref{tab:S:parse,tab:S:align} представлены примеры использования опции
форматирования чисел \texttt{S}, предоставляемой пакетом \texttt{siunitx}.

\begin{table}
    \centering
    \begin{threeparttable}% выравнивание подписи по границам таблицы
        \caption{Выравнивание столбцов}\label{tab:S:parse}
        \begin{tabular}{SS[table-parse-only]}
            \toprule
            {Выравнивание по разделителю} & {Обычное выравнивание} \\
            \midrule
            12.345                        & 12.345                 \\
            6,78                          & 6,78                   \\
            -88.8(9)                      & -88.8(9)               \\
            4.5e3                         & 4.5e3                  \\
            \bottomrule
        \end{tabular}
    \end{threeparttable}
\end{table}

\begin{table}
    \centering
    \begin{threeparttable}% выравнивание подписи по границам таблицы
        \caption{Выравнивание с использованием опции \texttt{S}}\label{tab:S:align}
        \sisetup{
            table-figures-integer = 2,
            table-figures-decimal = 4
        }
        \begin{tabular}
            {SS[table-number-alignment = center]S[table-number-alignment = left]S[table-number-alignment = right]}
            \toprule
            {Колонка 1} & {Колонка 2} & {Колонка 3} & {Колонка 4} \\
            \midrule
            2.3456      & 2.3456      & 2.3456      & 2.3456      \\
            34.2345     & 34.2345     & 34.2345     & 34.2345     \\
            56.7835     & 56.7835     & 56.7835     & 56.7835     \\
            90.473      & 90.473      & 90.473      & 90.473      \\
            \bottomrule
        \end{tabular}
    \end{threeparttable}
\end{table}

\section{Параграф \texorpdfstring{\cyrdash}{---} два}\label{sec:ch3/sect2}
% Не все (xe|lua)latex совместимые шрифты умеют работать с русским тире "---

Некоторый текст.

\section{Параграф с подпараграфами}\label{sec:ch3/sect3}

\subsection{Подпараграф \texorpdfstring{\cyrdash}{---} один}\label{subsec:ch3/sect3/sub1}

Некоторый текст.

\subsection{Подпараграф \texorpdfstring{\cyrdash}{---} два}\label{subsec:ch3/sect3/sub2}

Некоторый текст.

\clearpage
